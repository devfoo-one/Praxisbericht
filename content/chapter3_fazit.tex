\chapter{Fazit}
\section{Praktikum und Studium}
Die im Bachelorstudiengang Medieninformatik erworbenen Grundlagen und Fähigkeiten ermöglichten es mir, meine Aufgaben zu erfüllen.
Besonderer Fokus lag hier auf praktischen Erfahrungen mit Java und Datenbanken, sowie grundlegenden Problemlösungsstrategien die mich befähigten ein vorhandenes Problem zu abstrahieren und dadurch zu lösen.
Ein Beispiel hierfür ist das Problem der Gruppenbildung von Straßen als Graphenproblem zu verstehen und dieses mit bestehenden Bibliotheken zu lösen.

Wünschenswerte weitere Inhalte im Studiengang wären meiner Meinung nach Grundlagen zu professionellen Java-Projekten (wie zum Beispiel die Verwendung von \textit{Maven}) und ein Abschnitt Statistik innerhalb der Mathematik-Module.

Während meines Praktikums beim DFKI wurde mir Zeit gegeben, mich mit den Grundlagen von Klassifikatoren zu befassen.
Hier wurde mein Interesse für Text-Klassifikation und Machine Learning geweckt.
Ich werde die hier erworbenen Kenntnissse in meiner Bachelorarbeit weiter vertiefen.

\section{Bewertung des Praktikums}

Das DFKI bot eine angenehme, offene Atmosphäre.
Als Praktikant fühlte ich mich nie unter Druck gesetzt etwas einfach nur schnell abarbeiten zu müssen, sondern es wurde viel Wert auf Verständnis und dem Erlernen neuer Fähigkeiten gelegt.
Ein Beispiel hierfür ist, dass mir die Zeit gegeben wurde, mich eine ganze Woche nur mit dem Thema Klassifikatoren zu befassen.
Sofern ein Interesse für Natural Langugage Processing und/oder Machine Learning vorliegt, bietet einem das DFKI sehr viel Fachwissen, Anregungen und Unterstützung.
Alle Kollegen waren jederzeit bereit, ihr Wissen zu teilen.
Mein Gesamteindruck ist sehr positiv.
Es war eine sehr interessante Zeit, in der ich sehr viel lernte und viele neue Fachgebiete kennenlernte.
