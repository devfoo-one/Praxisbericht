\chapter{Einleitung}
\label{sec:intro}
\section{Vorstellung des Praktikumsbetriebes}
\label{sec:intro:dfki}
Das Deutsche Forschungszentrum für Künstliche Intelligenz GmbH, im folgenden DFKI genannt, wurde 1988 gegründet.
Es unterhält Standorte in Kaiserslautern, Saarbrücken, Bremen und ein Projektbüro in Berlin.
Mit seinen 478 Mitarbeiten sowie 337 studentischen Mitarbeiten erforscht und entwickelt das DFKI innovative Softwaretechnologien auf der Basis von Methoden der Künstlichen Intelligenz.
Die notwendigen Gelder werden durch Ausschreibungen öffentlicher Fördermittelgeber wie der Europäischen Union, dem Bundesministerium für Bildung und Forschung (BMBF), dem Bundesministerium für Wirtschaft und Technologie (BMWi), den Bundesländern und der Deutschen Forschungsgemeinschaft (DFG) sowie durch Entwicklungsaufträge aus der Industrie akquiriert.\footnote{http://www.dfki.de/web/ueber}
\\\\
Ich absolvierte mein Praktikum innerhalb der \textit{Forschungsgruppe Sprachtechnologie}, einer von 15 Forschungsgruppen\footnote{http://www.dfki.de/web/ueber/orgaeinheiten} des DFKI, im Projektbüro Berlin. Die Gruppe wird geleitet durch Prof. Dr. Hans Uszkoreit.\footnote{http://www.dfki.de/lt/}.
\\\\
Meine Aufgabengebiete konzentrierten sich um das Projekt \textit{''SD4M - Smart Data for Mobility''}. Das DFKI ist hier Teil eines Konsortiums aus 5 Partnern unter der Konsortialführung der \textit{DB Systel GmbH}. Das Projekt \textit{SD4M} wird in Abschnitt \ref{sec:main:overview} auf Seite \pageref{sec:main:overview} näher erläutert.
\section{Weg zur Praktikumsstelle}
\label{sec:intro:motivation}
foo.