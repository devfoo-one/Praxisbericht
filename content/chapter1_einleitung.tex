\chapter{Einleitung}
\label{sec:intro}
\section{Vorstellung des Praktikumsbetriebes}
\label{sec:intro:dfki}
Das Deutsche Forschungszentrum für Künstliche Intelligenz GmbH, im folgenden DFKI genannt, wurde 1988 gegründet.
Es unterhält Standorte in Kaiserslautern, Saarbrücken, Bremen und ein Projektbüro in Berlin.
Mit seinen 478 Mitarbeitern sowie 337 studentischen Mitarbeitern erforscht und entwickelt das DFKI innovative Softwaretechnologien auf der Basis von Methoden der Künstlichen Intelligenz.
Die notwendigen Gelder werden durch Ausschreibungen öffentlicher Fördermittelgeber wie der Europäischen Union, dem Bundesministerium für Bildung und Forschung (BMBF), dem Bundesministerium für Wirtschaft und Technologie (BMWi), den Bundesländern und der Deutschen Forschungsgemeinschaft (DFG) sowie durch Entwicklungsaufträge aus der Industrie akquiriert.\footnote{http://www.dfki.de/web/ueber}
\\\\
Ich absolvierte mein Praktikum innerhalb der \textit{Forschungsgruppe Sprachtechnologie}, einer von 15 Forschungsgruppen\footnote{http://www.dfki.de/web/ueber/orgaeinheiten} des DFKI, im Projektbüro Berlin. Die Gruppe wird geleitet durch Prof. Dr. Hans Uszkoreit.\footnote{http://www.dfki.de/lt/}.
\\\\
Meine Aufgabengebiete konzentrierten sich um das Projekt \textit{''SD4M - Smart Data for Mobility''}. Das DFKI ist hier Teil eines Konsortiums aus 5 Partnern unter der Konsortialführung der \textit{DB Systel GmbH}.\footnote{http://sd4m.net/#konsortium}. 
Das Projekt \textit{SD4M} wird in Abschnitt \ref{sec:main:overview:sd4m} auf Seite \pageref{sec:main:overview:sd4m} näher erläutert.
\section{Weg zur Praktikumsstelle}
\label{sec:intro:wegZurPraktikumsstelle}
Herr Prof. Dr. habil. Alexander Löser aus dem Fachbereich VI der Beuth Hochschule für Technik Berlin machte mich auf den Praktikumsplatz aufmerksam. Durch seine Mitarbeit in Projekten beim DFKI Projektbüro Berlin hatte er wahrgenommen, dass Bedarf und Interesse an Praktikanten und studentischen Mitarbeitern besteht und mich benachrichtigt.
Ich habe mich daraufhin auf der Website des DFKI über die aktuellen Projekte informiert.
Da ich mich sehr für Datenintegration interessiere und vor meinem Studium bereits Berufserfahrung auf diesem Gebiet gesammelt habe, fand ich das Projekt \textit{SD4M - Smart Data for Mobility} sehr interessant.
Es umfasst zwei Themenkomplexe: Zum einen die Verknüpfung unterschiedlicher Datenquellen und zum anderen Methoden des \textit{Natural Language Processings} bzw. des \textit{Text Minings}.
Das Thema Text Mining wurde kurz im Modul Datenbanksysteme im zweiten Semester angeschnitten und hatte mich auch sehr interessiert.
Nach einem persönlichen Gespräch mit Herr Uszkoreit, in dem ich mein Interesse für das Projekt darlegen konnte, kam es zur Vertragsunterzeichnung.